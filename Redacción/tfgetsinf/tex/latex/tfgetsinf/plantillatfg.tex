%%%%%%%%%%%%%%%%%%%%%%%%%%%%%%%%%%%%%%%%%%%%%%%%%%%%%%%%%%%%%%%%%%%%%%%%%%%%%%%
%                       CARREGA DE LA CLASSE DE DOCUMENT                      %
%                                                                             %
% Les opcions admissibles son:                                                %
%      12pt / 11pt            (cos dels tipus de lletra; no feu servir 10pt)  %
%                                                                             %
% catalan/spanish/english     (llengua principal del treball)                 %
%                                                                             % 
% french/italian/german...    (si necessiteu fer servir alguna altra llengua) %
%                                                                             %
% listoffigures               (El document inclou un Index de figures)        %
% listoftables                (El document inclou un Index de taules)         %
% listofquadres               (El document inclou un Index de quadres)        %
% listofalgorithms            (El document inclou un Index d'algorismes)      %
%                                                                             %
%%%%%%%%%%%%%%%%%%%%%%%%%%%%%%%%%%%%%%%%%%%%%%%%%%%%%%%%%%%%%%%%%%%%%%%%%%%%%%%

\documentclass[11pt,spanish,listoffigures,listoftables]{tfgetsinf}

%%%%%%%%%%%%%%%%%%%%%%%%%%%%%%%%%%%%%%%%%%%%%%%%%%%%%%%%%%%%%%%%%%%%%%%%%%%%%%%
%                     CODIFICACIO DEL FITXER FONT                             %
%                                                                             %
%    windows fa servir normalment 'ansinew'                                   %
%    amb linux es possible que siga 'latin1' o 'latin9'                       %
%    Pero el mes recomanable es fer servir utf8 (unicode 8)                   %
%                                          (si el vostre editor ho permet)    % 
%%%%%%%%%%%%%%%%%%%%%%%%%%%%%%%%%%%%%%%%%%%%%%%%%%%%%%%%%%%%%%%%%%%%%%%%%%%%%%%

\usepackage[utf8]{inputenc} 

%%%%%%%%%%%%%%%%%%%%%%%%%%%%%%%%%%%%%%%%%%%%%%%%%%%%%%%%%%%%%%%%%%%%%%%%%%%%%%%
%                        ALTRES PAQUETS I DEFINICIONS                         %
%                                                                             %
% Carregueu aci els paquets que necessiteu i declareu les comandes i entorns  %
%                                          (aquesta seccio pot ser buida)     %
%%%%%%%%%%%%%%%%%%%%%%%%%%%%%%%%%%%%%%%%%%%%%%%%%%%%%%%%%%%%%%%%%%%%%%%%%%%%%%%



%%%%%%%%%%%%%%%%%%%%%%%%%%%%%%%%%%%%%%%%%%%%%%%%%%%%%%%%%%%%%%%%%%%%%%%%%%%%%%%
%                        DADES DEL TREBALL                                    %
%                                                                             %
% titol, alumne, tutor i curs academic                                        %
%%%%%%%%%%%%%%%%%%%%%%%%%%%%%%%%%%%%%%%%%%%%%%%%%%%%%%%%%%%%%%%%%%%%%%%%%%%%%%%

\title{Estrategias de aprendizaje automático aplicadas a videojuegos}
\author{Adrián Valero Gimeno}
\tutor{Vicent Botti Navarro \\
Javier Palanca }
\curs{2018-2019}

%%%%%%%%%%%%%%%%%%%%%%%%%%%%%%%%%%%%%%%%%%%%%%%%%%%%%%%%%%%%%%%%%%%%%%%%%%%%%%%
%                     PARAULES CLAU/PALABRAS CLAVE/KEY WORDS                  %
%                                                                             %
% Independentment de la llengua del treball, s'hi han d'incloure              %
% les paraules clau i el resum en els tres idiomes                            %
%%%%%%%%%%%%%%%%%%%%%%%%%%%%%%%%%%%%%%%%%%%%%%%%%%%%%%%%%%%%%%%%%%%%%%%%%%%%%%%

\keywords{?????????????????} % Paraules clau 
         {Inteligencia artificial, aprendizaje, automatico, videojuegos, OpenAI, hiperparámetros}              % Palabras clave
         {?????, ????? ?????, ?????????????}        % Key words

%%%%%%%%%%%%%%%%%%%%%%%%%%%%%%%%%%%%%%%%%%%%%%%%%%%%%%%%%%%%%%%%%%%%%%%%%%%%%%%
%                              INICI DEL DOCUMENT                             %
%%%%%%%%%%%%%%%%%%%%%%%%%%%%%%%%%%%%%%%%%%%%%%%%%%%%%%%%%%%%%%%%%%%%%%%%%%%%%%%

\begin{document}

%%%%%%%%%%%%%%%%%%%%%%%%%%%%%%%%%%%%%%%%%%%%%%%%%%%%%%%%%%%%%%%%%%%%%%%%%%%%%%%
%              RESUMS DEL TFG EN VALENCIA, CASTELLA I ANGLES                  %
%%%%%%%%%%%%%%%%%%%%%%%%%%%%%%%%%%%%%%%%%%%%%%%%%%%%%%%%%%%%%%%%%%%%%%%%%%%%%%%

\begin{abstract}
???? (escribir más adelante)
\end{abstract}
\begin{abstract}[spanish]
????
\end{abstract}
\begin{abstract}[english]
????
\end{abstract}

%%%%%%%%%%%%%%%%%%%%%%%%%%%%%%%%%%%%%%%%%%%%%%%%%%%%%%%%%%%%%%%%%%%%%%%%%%%%%%%
%                              CONTINGUT DEL TREBALL                          %
%%%%%%%%%%%%%%%%%%%%%%%%%%%%%%%%%%%%%%%%%%%%%%%%%%%%%%%%%%%%%%%%%%%%%%%%%%%%%%%

\mainmatter

%%%%%%%%%%%%%%%%%%%%%%%%%%%%%%%%%%%%%%%%%%%%%%%%%%%%%%%%%%%%%%%%%%%%%%%%%%%%%%%
%                                  INTRODUCCIO                                %
%%%%%%%%%%%%%%%%%%%%%%%%%%%%%%%%%%%%%%%%%%%%%%%%%%%%%%%%%%%%%%%%%%%%%%%%%%%%%%%

\chapter{Introduccion}

El aprendizaje automático o “Machine Learning” ha sido durante los últimos años el foco de muchísima investigación, debido a su gran potencial en la aplicación a problemas del mundo moderno. En los últimos años, proyectos como AlphaGo o la supercomputadora de google Deep Mind han conseguido hacer grandes avances en juegos de gran dificultad, siendo los agentes desarrollados capaces de competir contra las mentes más experimentadas del tradicional juego de mesa Go. 
Hoy en día se están consiguiendo hacer grandes avances en el campo, debido a la investigación en nuevas técnicas como la combinación de redes neuronales tradicionales con otros métodos como el Deep Learning. Por el momento, Google está liderando este nuevo movimiento, con su supercomputadora DeepMind. Esta computadora fue capaz de desarrollar AlphaGo, una inteligencia artificial capaz de ganar a las mentes más experimentadas del juego tradicional chino Go. Éste juego, considerado uno de los más difíciles del mundo, tendría sobre $10^{172}$ configuraciones distintas - un número mayor que el número estimado de átomos existentes en el universo -, haciéndolo extraordinariamente más complejo que juegos como el ajedrez.


\section{Motivaci\'on}

????? ????????????? ????????????? ????????????? ????????????? ????????????? 

\section{Objetivos}

El propósito de este TFG consistirá en el estudio de las diferentes técnicas existentes de aprendizaje automático aplicadas a sencillos videojuegos en 2 dimensiones, los cuales tendrán un número limitado de acciones a realizar. Se pretende además, realizar implementaciones propias de las diferentes técnicas descritas durante el resto del documento. \par 

Otro de los objetivos consistirá en el estudio y realización de pruebas sobre distintos entornos predefinidos de juegos arcade, haciendo uso de librerías que permiten el acceso a dicho tipo de juegos como la herramienta OpenAI desarrollada por Google, o las distintas librerías de Python relacionadas con la creación de entornos de redes neuronales y, más concretamente, la manipulación de hiperparámetros para encontrar los mejores resultados para cada uno de los entornos estudiados. \par 

Por último, resultaría interesante aplicar estas nuevas técnicas estudiadas en la resolución de algún videojuego de mayor complejidad. Se plantea, por lo tanto, un estudio de los algoritmos más destacables estudiados en algún juego complejo como puede ser Montezuma's Revenge, un juego de la atari 2600 estrenado en el año 1983.


\section{Metodología}

Para la realización de este trabajo se pretende realizar una investigación en el campo de la inteligencia artificial, concretamente en el desarrollo de algoritmos de aprendizaje aplicados a entornos estocásticos. Para ello, se investigarán técnicas como el aprendizaje por refuerzo o Q-Learning, algoritmos evolutivos, o el uso de redes neuronales. \par

Para complementar nuestra base de conocimientos, nos basaremos en artículos de investigación publicados por entidades como Google DeepMind, o el Massachussets Institute of Technology, en busca de ideas y nuevas técnicas para su posterior aplicación en nuestro trabajo.


\section{Estructura de la memoria}

????? ????????????? ????????????? ????????????? ????????????? ????????????? 

%\section{Notes bibliografiques} %%%%% Opcional

%????? ????????????? ????????????? ????????????? ????????????? ?????????????

%%%%%%%%%%%%%%%%%%%%%%%%%%%%%%%%%%%%%%%%%%%%%%%%%%%%%%%%%%%%%%%%%%%%%%%%%%%%%%%
%                         CAPITOLS (tants com calga)                          %
%%%%%%%%%%%%%%%%%%%%%%%%%%%%%%%%%%%%%%%%%%%%%%%%%%%%%%%%%%%%%%%%%%%%%%%%%%%%%%%


\chapter{Estado del arte. La situación del aprendizaje automático en la actualidad}

??????????????????????????????????????????????????????????

\section{?? ???? ???? ? ?? ??}

????? ????????????? ????????????? ????????????? ????????????? ?????????????

\chapter{??? ???? ??????}

????? ????????????? ????????????? ????????????? ????????????? ????????????? 

\section{?? ???? ???? ? ?? ??}

????? ????????????? ????????????? ????????????? ????????????? ?????????????

%%%%%%%%%%%%%%%%%%%%%%%%%%%%%%%%%%%%%%%%%%%%%%%%%%%%%%%%%%%%%%%%%%%%%%%%%%%%%%%
%                                 CONCLUSIONS                                 %
%%%%%%%%%%%%%%%%%%%%%%%%%%%%%%%%%%%%%%%%%%%%%%%%%%%%%%%%%%%%%%%%%%%%%%%%%%%%%%%

\chapter{Conclusions}

????? ????????????? ????????????? ????????????? ????????????? ????????????? 

%%%%%%%%%%%%%%%%%%%%%%%%%%%%%%%%%%%%%%%%%%%%%%%%%%%%%%%%%%%%%%%%%%%%%%%%%%%%%%%
%                                BIBLIOGRAFIA                                 %
%%%%%%%%%%%%%%%%%%%%%%%%%%%%%%%%%%%%%%%%%%%%%%%%%%%%%%%%%%%%%%%%%%%%%%%%%%%%%%%

\begin{thebibliography}{10}

%%%%%%%%%%%%%%%%%%%%%%%%%%%%%%%%%%%%%%%%%%%%%%%%%%%%%%%%%%%%%%%%%%%%%%%%%%%%%%%
% MODEL D'ARTICLE                                                             %
%%%%%%%%%%%%%%%%%%%%%%%%%%%%%%%%%%%%%%%%%%%%%%%%%%%%%%%%%%%%%%%%%%%%%%%%%%%%%%%
\bibitem{light}
   Jennifer~S. Light.
   \newblock When computers were women.
   \newblock \textit{Technology and Culture}, 40:3:455--483, juliol, 1999.

%%%%%%%%%%%%%%%%%%%%%%%%%%%%%%%%%%%%%%%%%%%%%%%%%%%%%%%%%%%%%%%%%%%%%%%%%%%%%%%
% MODEL DE LLIBRE                                                             %
%%%%%%%%%%%%%%%%%%%%%%%%%%%%%%%%%%%%%%%%%%%%%%%%%%%%%%%%%%%%%%%%%%%%%%%%%%%%%%%
\bibitem{ifrah}
   Georges Ifrah.
   \newblock \textit{Historia universal de las cifras}.
   \newblock Espasa Calpe, S.A., Madrid, sisena edició, 2008.

%%%%%%%%%%%%%%%%%%%%%%%%%%%%%%%%%%%%%%%%%%%%%%%%%%%%%%%%%%%%%%%%%%%%%%%%%%%%%%%
% MODEL D'URL                                                                 %
%%%%%%%%%%%%%%%%%%%%%%%%%%%%%%%%%%%%%%%%%%%%%%%%%%%%%%%%%%%%%%%%%%%%%%%%%%%%%%%
\bibitem{WAR}
   Comunicat de premsa del Departament de la Guerra, 
   emés el 16 de febrer de 1946. 
   \newblock Consultat a 
   \url{http://americanhistory.si.edu/comphist/pr1.pdf}.
   
\bibitem{Mnih}
	Volodymyr Mnih, Adrià Puigdomènech Badia, Mehdi Mirza, Alex Graves, Tim Harley, Timothy P. Lillicrap, David Silver, Koray Kavukcuoglu.
	\newblock Asynchronous Methods for Deep Reinforcement Learning,
	febrero de 2016.
	\newblock \textit {Google DeepMind, Montreal Institute for Learning Algorithms (MILA), University of Montreal}
	
\bibitem{Tokic}
	Michel Tokic
	\newblock Adaptive $\varepsilon$-greedy Exploration in Reinforcement Learning Based on Value Differences.
	\newblock \textit {Institute of Applied Research, University of Applied Sciences Ravensburg-Weingarten, 88241 Weingarten, Germany}
	
\bibitem{Wu}
	Jianxin Wu
	\newblock Introduction to Convolutional Neural Networks,
	mayo de 2017.
	\newblock \textit {LAMDA Group \newblock National Key Lab for Novel Software Technology. Nanjing University, China}
	
\bibitem{Schaul}
	Tom Schaul, John Quan, Ioannis Antonoglou, David Silver
	\newblock Prioritized Experience Replay,
	febrero de 2016.
	\newblock \textit {Google DeepMind}

\end{thebibliography}
\cleardoublepage

%%%%%%%%%%%%%%%%%%%%%%%%%%%%%%%%%%%%%%%%%%%%%%%%%%%%%%%%%%%%%%%%%%%%%%%%%%%%%%%
%                           APÈNDIXS  (Si n'hi ha!)                           %
%%%%%%%%%%%%%%%%%%%%%%%%%%%%%%%%%%%%%%%%%%%%%%%%%%%%%%%%%%%%%%%%%%%%%%%%%%%%%%%

\APPENDIX

%%%%%%%%%%%%%%%%%%%%%%%%%%%%%%%%%%%%%%%%%%%%%%%%%%%%%%%%%%%%%%%%%%%%%%%%%%%%%%%
%                         LA CONFIGURACIO DEL SISTEMA                         %
%%%%%%%%%%%%%%%%%%%%%%%%%%%%%%%%%%%%%%%%%%%%%%%%%%%%%%%%%%%%%%%%%%%%%%%%%%%%%%%

\chapter{Configuració del sistema}

????? ????????????? ????????????? ????????????? ????????????? ?????????????

\section{Fase d'inicialització}

????? ????????????? ????????????? ????????????? ????????????? ?????????????

\section{Identificació de dispositius}

????? ????????????? ????????????? ????????????? ????????????? ?????????????

%%%%%%%%%%%%%%%%%%%%%%%%%%%%%%%%%%%%%%%%%%%%%%%%%%%%%%%%%%%%%%%%%%%%%%%%%%%%%%%
%                               ALTRES  APÈNDIXS                              %
%%%%%%%%%%%%%%%%%%%%%%%%%%%%%%%%%%%%%%%%%%%%%%%%%%%%%%%%%%%%%%%%%%%%%%%%%%%%%%%


\chapter{??? ???????????? ????}

????? ????????????? ????????????? ????????????? ????????????? ????????????? 



%%%%%%%%%%%%%%%%%%%%%%%%%%%%%%%%%%%%%%%%%%%%%%%%%%%%%%%%%%%%%%%%%%%%%%%%%%%%%%%
%                              FI DEL DOCUMENT                                %
%%%%%%%%%%%%%%%%%%%%%%%%%%%%%%%%%%%%%%%%%%%%%%%%%%%%%%%%%%%%%%%%%%%%%%%%%%%%%%%

\end{document}
